\renewcommand{\abstractname}{Περίληψη}

\begin{abstract}
Τα τελευταία χρόνια το cloud computing αποτελεί ένα σημαντικό κεφάλαιο στη
σύγχρονη επιστήμη υπολογιστών. Η κύρια τεχνολογία που χρησιμοποιείται,
	προκειμένου να μπορεί να υποστηριχθεί το cloud computing είναι αυτή της
	εικονικοποίησης. Με αυτό τον τρόπο ένα φυσικό μηχάνημα, μπορεί να
	φιλοξενήσει πολλά εικονικά μηχανήματα, κάθε ένα από τα οποία αποτελεί
	έναν αυτοδύναμο υπολογιστή. Ωστόσο, αποτελεί συχνό φαινόμενο οι
	εικονικές αυτές μηχανές να χρησιμοποιούνται για την εκτέλεση μίας και
	μόνο εφαρμογής. Αυτό έχει ως αποτέλεσμα, να χαραμείζονται πόροι σε
	ενέργειες που δε χρειάζονται από την εφαρμογή, αλλά είναι απαραίτητες
	για το λειτουργικό σύστημα στο οποίο τρέχουν αυτές οι εφαρμογές. 
	
Μία νεότερη τάση για την υποστήριξη του cloud computing είναι τα
	containers, τα οποία όμως παρουσιάζουν αρκετά σημαντικά ζητήματα που
	αφορούν την ασφάλεια. Ένα από τα ζητήματα αυτά είναι, εκείνο της
	απομόνωσης το οποίο αναγκάζει σε αρκετές περιπτώσεις να οδηγεί στη χρήση
	εικονικών μηχανών για τη φιλοξενία των containers, χάνοντας αρκετά από
	τα πλεονεκτήματα τους.  

Μία ακόμη προσέγγιση στο θέμα είναι οι unikernels. Πρόκειται για μία εικόνα
	εικονικής μηχανής, με ένα μόνο address space το οποίο κατασκευάζεται από
	library operating systems και είναι ειδικευμένο για μία συγκεκριμένη
	εφαρμογή. Πιο απλά, περιέχει τον κώδικα της εφαρμογής και ακριβώς ό,τι
	κομμάτι του λειτουργικού συστήματος χρειάζεται η εφαρμογή για να
	λειτουργήσει η διεργασία (drivers, βιβλιοθήκες, κ.λ.π.), ενοποιημένα 
	σαν ένα	αυτόνομο πρόγραμμα που μπορεί να τρέξει ως εικονική μηχανή.
	Εν τούτοις, ένα πρόβλημα είναι ότι οι unikernels υποστηρίζουν μία και
	μόνο διεργασία, με αποτέλεσμα να μην μπορούν εφαρμογές υλοποιημένες για
	τα πλήρη λειτουργικά συστήματα να εκτελεστούν σε unikernels. 

Σκοπός, λοιπόν, αυτής της εργασίας είναι η υλοποίηση ενός μηχανισμού που θα
	επιτρέπει σε εφαρμογές με περισσότερες από μία διεργασίες να μπορούν να
	εκτελεστούν και σε unikernels. Επιπλέον, υλοποιείται και ένας απλός
	μηχανισμός για διαδιεργασιακή επικοινωνία, στα πρότυπα του pipe. 
\vspace{2ex}

Λέξεις-Κλειδιά: εικονικοποίηση, εικονικές μηχανές, ενδοεπικοινωνία εικονικών μηχανών, unikernel, kvm, QEMU
\end{abstract}

\renewcommand{\abstractname}{Abstract}

\begin{abstract}
	Abstract
\end{abstract}

