\renewcommand{\abstractname}{Περίληψη}

\begin{abstract}
Τα τελευταία χρόνια το cloud computing αποτελεί ένα σημαντικό κεφάλαιο στη
σύγχρονη επιστήμη υπολογιστών. Η κύρια τεχνολογία που χρησιμοποιείται,
	προκειμένου να μπορεί να υποστηριχθεί το cloud computing είναι αυτή της
	εικονικοποίησης. Με αυτό τον τρόπο ένα φυσικό μηχάνημα, μπορεί να
	φιλοξενήσει πολλά εικονικά μηχανήματα, κάθε ένα από τα οποία αποτελεί
	έναν αυτοδύναμο υπολογιστή. Ωστόσο, αποτελεί συχνό φαινόμενο οι
	εικονικές αυτές μηχανές να χρησιμοποιούνται για την εκτέλεση μίας και
	μόνο εφαρμογής. Αυτό έχει ως αποτέλεσμα, να χαραμείζονται πόροι σε
	ενέργειες που δε χρειάζονται από την εφαρμογή, αλλά είναι απαραίτητες
	για το λειτουργικό σύστημα στο οποίο τρέχουν αυτές οι εφαρμογές. 
	
Μία νεότερη τάση για την υποστήριξη του cloud computing είναι τα
	containers, που προσφέρουν ελαφρύτερη εικονικοποίηση με γρηγορους
	χρόνους εκτέλεσης, μικρή κατανάλωση μνήμης και άλλα πλεονεκτήματα. Από
	την άλλη, παρουσιάζουν αρκετά σημαντικά ζητήματα που
	αφορούν την ασφάλεια. Ένα από τα ζητήματα αυτά είναι, εκείνο της
	απομόνωσης το οποίο αναγκάζει σε αρκετές περιπτώσεις να οδηγεί στη χρήση
	εικονικών μηχανών για τη φιλοξενία των containers, χάνοντας αρκετά από
	τα πλεονεκτήματα τους.  

Μία ακόμη προσέγγιση στο θέμα είναι οι unikernels. Πρόκειται για μία εικόνα
	εικονικής μηχανής, με ένα μόνο address space το οποίο κατασκευάζεται από
	library operating systems και είναι ειδικευμένο για μία συγκεκριμένη
	εφαρμογή. Πιο απλά, περιέχει τον κώδικα της εφαρμογής και ακριβώς ό,τι
	κομμάτι του λειτουργικού συστήματος χρειάζεται η εφαρμογή για να
	λειτουργήσει η διεργασία (drivers, βιβλιοθήκες, κ.λ.π.), ενοποιημένα 
	σαν ένα	αυτόνομο πρόγραμμα που μπορεί να τρέξει ως εικονική μηχανή.
	Οι unikernels καταφέρνουν να έχουν γρήγορους χρόνους εκκίνησης και μικρή
	κατανάλωση μνήμης, χωρίς να θυσιάζεται η ασφάλεια
	Εν τούτοις, ένα πρόβλημα είναι ότι οι unikernels υποστηρίζουν μία και
	μόνο διεργασία, με αποτέλεσμα να μην μπορούν εφαρμογές με παραπάνω από
	μία διεργασίες να εκτελεστούν σε unikernels. 

Σκοπός, λοιπόν, αυτής της εργασίας είναι η υλοποίηση ενός μηχανισμού που θα
	επιτρέπει σε εφαρμογές με περισσότερες από μία διεργασίες να μπορούν να
	εκτελεστούν και σε unikernels. Επιπλέον, υλοποιείται και ένας απλός
	μηχανισμός για επικοινωνία μεταξύ των εικονικών μηχανών, στα πρότυπα του pipe. 
\vspace{2ex}

Λέξεις-Κλειδιά: εικονικοποίηση, εικονικές μηχανές, ενδοεπικοινωνία εικονικών μηχανών, unikernel, kvm, QEMU
\end{abstract}

\renewcommand{\abstractname}{Abstract}

\begin{abstract}
In recent years cloud computing is one important chapter of modern commputer
	science. The main technology used in order to support cloud computing is
	virtualization. Virtualization makes possible for a physical machine to
	host many virtual machines, each one of which is a self-sufficient
	computer. However, virtual machines are often used to execute a single
	application. As a result, resources are devoted to actions that are not
	needed by the application, but they are necessary for the operating
	system in which the applications run.

An another technology to support cloud computing is containers which offer
	lightweight virtualization, fast instantiation times and small
	per-instance 
	memory footprints among other features. On the other hand, containers
	have several important security issues. Isolation is one of these issues
	, which in several cases leads to the use of virtual machines to host
	the containers, losing several of their advantages.

A further approach in cloud computing is unikernerls. Unikernels are
	specialised, single-address-space machine images constructed by using
	library operating systems and are specialised for one application. In
	somewhat simplified terms, unikernels consist of the application's
	source code and the parts of an opperating system that are necessary for
	the proccess to run (drivers, libraries, etc.) consolidated as a
	stand-alone virtual machine. The Unikernels manage to have fast
	instantiation times, small memory footprints, without sacrificing
	security. However, one of the problems of unikernels is that they are
	single-proccess and as a result multi-proccess applcations are not able
	to run on unikernels. 

The purpose of this thesis is to implement a mechanism that will enable the
	execution of multi-proccess applications on unikernels. Furthermore, a
	pipe-like mechanism for inter-vm communication is impemented.
\vspace{2ex}

Keywords: virtualization, virtual machines, inter-vm communication, unikernel, kvm, QEMU
\end{abstract}

