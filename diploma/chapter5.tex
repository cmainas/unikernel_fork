\chapter{Επίλογος}
\label{chap:conclusion}
Το cloud computing, είτε αυτό αφορά σε κατανεμημένα - γεωγραφικά - περιβάλλοντα
(grid computing) είτε αφορά υπολογιστικά κέντρα με συστοιχίες (clusters)
υπολογιστών, βρίσκεται αναπόφευκτα στο επίκεντρο του ενδιαφέροντος στις μέρες
μας. Η εικονικοποίηση, ως η βασική τεχνολογία πίσω από το cloud computing, δίνει
τη δυνατότητα να αξιοποιήσουμε όσο το δυνατόν καλύτερα τους πόρους του
συστήματος και επιτρέπει τη μεταφορά ενός ολόκληρου συστήματος από ένα φυσικό
υπολογιστικό σύστημα σε ένα άλλο. Χάρις αυτές τις τεχνολογίες, πετυχαίνουμε την
εκτέλεση περισσότερων εφαρμογών και με περισσότερη ασφάλεια στο ίδιο
υπολογιστικό σύστημα. Σε αυτό το πλαίσιο αναπτύχθηκαν νέες τεχνολογίες
εικονικοποίησης, όπως η εικονικοποίηση σε επίπεδο λειτουργικού συτήματος
(containers), που αφαιρούν το κόστος της εκτέλεσης ενός hypervisor, αλλά
μειώνοντας την απομόνωση μεταξύ των containers.

%%Ωστόσο η εκτέλεση λειτουργιών και εφαρμογών στα πλαίσια
%%ενός γενικού σκοπού λειτουργικού συστήματος μπορεί μεν να προσφέρει τα
%%πλεονεκτήματα της ευχρηστίας ή της ευκολότερης ανάπτυξης της, ωστόσο έρχεται με
%%το κόστος των πρόσθετων λειτουργιών που προσφέρει ακριβώς το γεγονός ότι αυτό
%%είναι γενικού σκοπού. 

Ακολουθώντας μία διαφορετική προσέγγιση, που θέλει τo λειτουργικό σύστημα να
είναι tailored στις ανάγκες μίας εφαρμογής και όχι το ανάποδο, μπορούμε να
επανασχεδιάσουμε τα περιβάλλοντα εκτέλεσης των εφαρμογών ώστε να περιλαμβάνουν
μόνο ό,τι πραγματικά χρειάζεται η εκτέλεσή της πάνω από το υλικό. Η προσέγγιση
αυτή καλύπτεται από τους unikernels, οι οποίοι αποτελούν ανεξάρτητες εικόνες μηχανές
ενιαίου χώρου διευθύνσεων ικανές να εκτελεστούν αυτόνομα, περιλαμβάνοντας όλα
και μόνο τα στοιχεία που χρειάζονται για να το πετύχουν αυτό. Ακόμη και σε αυτή
τη προσέγγιση υπάρχουν διαφορετικές οπτικές. Η επαναδιαμόρφωση του
νεφοϋπολογιστικού περιβάλλοντος εκτέλεσης της εφαρμογής μπορεί είτε να σημαίνει
την εξαρχής επανασχεδίαση και ανάπτυξη των στοιχείων που παραδοσιακά παρείχε το
λειτουργικό σύστημα είτε την επαναχρησιμοποίηση ήδη υπαρκτών.

Η δεύτερη περίπτωση φέρει το πλεονέκτημα του ευκολότερου \EN{porting} των υπαρκτών
εφαρμογών που ακολουθούν τα παραδοσιακά πρότυπα (POSIX) καθώς και την προφανή
ανάγκη επαναπρογραμματισμού των βιβλιοθηκών και των drivers. Ωστόσο ο
κατακερματισμός ενός λειτουργικού συστήματος σε αυτόνομες επαναχρησιμοποιούμενες
μονάδες δεν είναι εύκολη υπόθεση, ιδιαίτερα αν αναλογιστούμε ότι κατά τον αρχικό
σχεδιασμό τους δεν υπήρχε μάλλον η ανάγκη και το κριτήριο της αποφυγής (κατά το
δυνατόν) των πολλαπλών αλληλοεξαρτήσεων. Ακόμα όμως και σε αυτό το πλαίσιο το
porting των εφαμρογών σε unikernels, δεν αποτελεί εύκολη διαδικασία, καθώς πολλές
λειτουργίες που προσφέρονται από τα συμβατικά λειτουργικά συστήματα δεν υπάρχουν
στα unikernel frameworks. Χαρακτηριστικό παράδειγμα αποτελούν οι multi-process
εφαρμογές, οι οποίες δεν μπορούν να εκτελεστούν σε κάποιο unikernel, καθώς
κανένα δεν υποστηρίζει περισσότερες από μία διεργασίες.

Θέλοντας να κάνουμε τα unikernel frameworks, περισσότερο φιλόξενα για τις ήδη
υπάρχουσες εφαρμογές, σχεδιάσαμε και υλοποιήσαμε δύο μηχανσιμούς. Έναν για την
επικοινωνία μεταξύ unikernels στα πλαίσια της κλήσης συστήματος pipe και ένα για
τη δημιουργία νέων unikernels, όπως δημιουργούνται νέες διεργασίες με την κλήση
fork. Σκεφτόμενοι τα unikernels ως διεργασίες και το hypervisor ως λειτουργικό
σύτημα, δημιουργήσαμε τους δύο παραπάνω μηχανισμούς, με στόχο να μοιάζουν με
τους αντίστοιχους μηχανισμούς στα συμβατικά λειτουργικά συστήματα.

Αρχικά παρουσιάστηκε στον αναγνώστη το θεωρητικό υπόβαθρο που ήταν απαραίτητο
για την κατανόηση των εννοιών που χρησιμοποιήθηκαν, αλλά και για την κατανόηση
της σχεδίασης και υλοποίησης των μηχανισμών. Στη συνέχεια, έγινε μία σύντομη
περιγραφή των unikernel frameworks που μελετήθηκαν σε αυτή την εργασία και
παρουσιάζονται κάποια κύρια χαρακτηριστικά τους. Έπειτα, παρουσιάσαμε το
σχεδιασμό των δύο μηχανισμών που υλοποιήσαμε. 

Αν και υπάρχουν library operating systems, που μπορούν να υποστηρίξουν
multi-process εφαρμογές, όπως το Graphene ~\cite{tsai2014cooperation} δεν υπήρχε
κάποιος ανάλογος μηχανισμός, που μεταχειρίζεται τα \EN{unikernels} ως διεργασίες.
Ωστόσο κατά τη διάρκεια της εκπόνησης της εργασίας, παρουσιάστηκε το Kylinx
~\cite{zhang2018kylinx}, που εισάγει την έννοια του pVM (process-like VM). Το
συγκεκριμένο λειτουργικό συτημα τρέχει μόνο πάνω από Xen hypervisor και πέρα από
την υποστήριξη της fork, παρέχει και ένα inter-pVM communication API, που
περιλαμβάνει τους μηχανη=ισμούς pipe, signal, message queue και shared memory. 
Επιπροσθέτως η αντιμετόπιση των unikernels ως διεργασίες παρουσιάστηκε και σε ένα νέο paper το Νοέμβριο του 2018 ~\cite{williams2018unikernels}.

\section{Μελλοντικές κατευθύνσεις}
Ξεκινώνοντας από το μηχανσιμό pipe, θα μπορούσαμε να τον επεκτείνουμε ώστε να
μπορεί να χρησιμοποιεί TCP sockets αντί γοα UDP, στο δεύτερο στάδιο υλοποίησης
του UDP. Επιπλέον ο μηχανισμός ivshmem, παρέχει υποστήριξη για την αποστολή
σημάτων
μεταξύ των εικονικών μηχανών. Αυτό το χαρακτηριστικό θα μπορούσε να
χρησιμοποιηθεί για την καλύτερη υλοποίηση του μηχανισμού, ώστε να μη βασίζεται
τόσο στην κοινή μνήμη. Επιπλέον το συγκεκριμένο χαρακτηριστικό μπορεί να
βοηθήσει στην υλοποίηση ενός μηχανσιμού signaling, παρόμοιο με αυτό που παρέχουν
τα συμβατικά λειτουργικά συστήματα. Τέλος αρκετά ενδιαφέρον θα ήταν να
συγκρίνουμε το συγκεκριμένο μηχανισμό και να τον βελτιστοποιήσουμε, ώστε να
πλησιάζει, όσο γίνεται, στον κλασσικό μηχανισμό pipe.

Από τη μεριά του μηχανισμού fork, θεωρούμε ότι χωράει αρκετές βελτιστοποιήσεις.
Ιδανικά θα θέλαμε ο χρόνος που απαιτείται, να πλησιάζει όσο το δυνατόν
περισσότερο σε αυτόν του fork σε ένα συμβατικό λειτουργικό σύστημα. Η ύπαρξη δύο
busy waits στην υλοποίηση δημιουργεί αυτόματα πεδίο για εξερεύνηση εναλλακτικών
επιλογών που θα επαλείφουν την καθυστέρηση αυτή. Επιπλέον, η χρήση του
μηχανισμού migration του QEMU, ίσως να μπορεί να αντικατασταθεί από ένα ταχύτερο
και με περισσότερες λειτουργίες μηχανισμό. Τέλος, ο συγκεκριμένος μηχανισμός
είναι υλοποιημένος για το QEMU και θα μπορούσε να υλοποιηθεί και σε άλλους
hyervisors, ή ακόμα και σε tenders όπως το solo5.


